\documentclass[12pt]{article}
\usepackage{amsmath}
\begin{document}
\begin{title}
    \centering
    {\LARGE\bfseries Exercise 2.3.1}
\end{title}\\

\vspace{10mm} %10mm vertical space
Show the following formulas by induction on n starting at n = 1.

\vspace{5mm} %5mm vertical space

b) $\sum_{i=1}^{n}i^2 = \frac{n(n+1)(2n+1)}{6}$\par

\vspace{5mm} %5mm vertical space

Basis case: LHS when n = 1 yields 1.\par

\vspace{5mm} %5mm vertical space

RHS when n = 1 is 1(1 + 1)(2 + 1)/6 = 6/6 = 1\par

LHS = RHS therefore s(n) is true when n is 1.\par

Inductive step: Assume that n $\geq$ 1 and that s(n) is true as proved above.\par I must prove that s(n + 1) is true which is:\par

\vspace{5mm} %5mm vertical space

$\sum_{i=1}^{n+1}i^2 = \frac{(n+1)(n+2)(2n+3)}{6}$   \hfil\hfil\hfil (2)\par

\vspace{5mm} %5mm vertical space

Replacing the LHS of the above equation with the following:\par

\vspace{5mm} %5mm vertical space

$\sum_{i=1}^{n+1}i^2 = (\sum_{i=1}^{n}i^2) + (n + 1)^2$\par

\vspace{5mm} %5mm vertical space

which can be rewritten as:\par

\vspace{5mm} %5mm vertical space

$\sum_{i=1}^{n+1}i^2 = \frac{n(n+1)(2n+1)}{6} + (n + 1)^2 = \frac{n(n+1)(2n+1)+6(n+1)^2}{6}$\par

\vspace{5mm} %5mm vertical space

\hfil $=  \frac{(n+1)(2n^2+n+6n+6)}{6}$\par

\vspace{5mm} %5mm vertical space

\hfil $=  \frac{(n+1)(2n^2+7n+6)}{6}$\par

\vspace{5mm} %5mm vertical space

\hfil $=  \frac{(n+1)(n + 2)(2n + 3)}{6}$\par

\vspace{5mm} %5mm vertical space

The final equation the right is equal to (2) proving that s(n + 1) is also\par true.
Therefore s(n) hold for all n $\geq$ 1
\end{document}